\oneLineTitle\\
\oneLineSubtitle\\
\\
Eli~Adelhult, Gustav~Bruhn, Carl~Forsinge, Axel~Larsson,\\ Hugo~Mårdbrink, Jonathan~Widén\\

Department of Computer Science and Engineering\\
Chalmers University of Technology and University of Gothenburg

\thispagestyle{plain}			% Supress header
\section*{Abstract}
ModMark is a new modular document markup language that aims to provide a balance between the simplicity of Markdown and the power of \LaTeX. It can be used both in an online playground and through the command-line. Modmark utilises the capabilities of WebAssembly and the WebAssembly System Interface, allowing for a highly extensible markup language that can be customised to fit the needs of specific use cases. ModMark allows users to easily create and manage custom packages that can be added to their documents to enhance functionality, while still maintaining a straightforward and intuitive syntax. The modularity of ModMark even allows users to write packages that add support for any output format they desire. The language is ergonomic enough to use for notes while also being powerful enough to produce this research paper.

\section*{Sammanfattning}
TODO
% KEYWORDS (MAXIMUM 10 WORDS)
\vfill
Keywords: Markup languages, WebAssembly, WASI

\newpage				% Create empty back of side
\thispagestyle{empty}
\mbox{}